%{
\documentclass{article}
\input{../hw1/commonheader}
\usepackage{comment}

\usepackage{enumitem}
\newenvironment{matlabc}{}{}
\newenvironment{octavec}{}{}
\excludecomment{matlabc}
%\newenvironment{matlabv}{\verbatim}{\endverbatim}
%\newenvironment{octavev}{\verbatim}{\endverbatim}


% ******** For code: ******************
\usepackage{listings}
%\usepackage[usenames,dvipsnames]{color}
% ******** Define the language: *******
\lstdefinelanguage{matlabfloz}{%
	alsoletter={...},%
	morekeywords={% 						% keywords
	break,case,catch,continue,elseif,else,end,for,function,global,%
	if,otherwise,persistent,return,switch,try,while,...,
        classdef,properties,methods},%
	comment=[l]\%,% 						% comments
	morecomment=[l]...,% 					% comments
	morestring=[m]',%   						% strings 
}[keywords,comments,strings]%
\lstdefinestyle{matlab}{language=matlabfloz,							
		keywordstyle=\color[rgb]{0,0,1},					
		commentstyle=\color[rgb]{0.133,0.545,0.133},	% comments
		stringstyle=\color[rgb]{0.627,0.126,0.941},	% strings
		numbersep=3mm, numbers=left, numberstyle=\tiny% number style
}
% ******** Set the style: **************
\lstset{style=matlab}

\begin{matlabc}
%}
% Start some Matlabe startup stuffs:
clear all
close all

% Record version of MATLAB/Octave
a = ver('octave');
if length(a) == 0
    a = ver('matlab');
end
fid = fopen('vers.tex', 'w');
fprintf(fid, [a.Name ' version ' a.Version]);
fclose(fid);

% Record computer architecture.
fid = fopen('computer.tex', 'w');
fprintf(fid, ['\\verb+' computer '+']);
fclose(fid);

% Record date of run.
fid = fopen('date.tex', 'w');
fprintf(fid, datestr(now, 'dd/mm/yyyy'));
fclose(fid);

%{
\end{matlabc}
\title{MATH 5410: Homework 2}             
\author{Andrew Pound}
%\date{\input{date.tex}}

\begin{document}
\maketitle

%Due on November 25: pg 166-168
%Exercises: 4,5,7(e),9


\section*{Problem 4}
Show that the first variation of $\delta J (y_0, h)$ satisfies the
homogeneity condition
\begin{equation*}
  \delta J(y_0, \alpha h) = \alpha\delta J(y_0, h), \quad \alpha \in
  \mathbb{R}. 
\end{equation*}

\subsection*{My Answer}
\begin{equation*}
  \begin{split}
    \delta J(y_0, \alpha h) 
    &= \frac{d}{d\epsilon}J(y_0 + \epsilon (\alpha h))|_{\epsilon =0}\\ 
    & = J'(y_0 + \epsilon(\alpha h))\cdot(\alpha h)|_{\epsilon =0}\\ 
    & = \alpha \underbrace{\left(hJ'(y_0 +  \epsilon(\alpha h))|_{\epsilon
        =0}\right)}_{\delta J(y_0,h)}\\
    &= \alpha\delta J(y_0, h). 
  \end{split}
\end{equation*}
In the thrid line above, we can say that the term in the parenthesis
is $\delta J(y_0,h)$, because when $\epsilon \rightarrow 0$, then the
term $\epsilon\alpha h \rightarrow 0$, just like the term $\epsilon h$
would.

\section*{Problem 5}
A functional $J: V \rightarrow \mathbb{R}$, where $V$ is a normed
linear space, is \emph{linear} if 
\begin{equation*}
  J(y_1 + y_2) = J(y_1) + J(y_2), \quad y_1,y_2 \in V
\end{equation*}
and
\begin{equation*}
  J(\alpha y_1) = \alpha J(y_1), \quad \alpha\in \mathbb{R}, \quad y_1
  \in V.
\end{equation*}
Which of the following functionals on $C^1[a,b]$ are linear?
\begin{enumerate}[label={(\alph*)}]
\item $ J(y) = \int_a^b yy' dx$.
\item[] Check additivity:
\begin{equation*}
    \begin{split}
      J(y_1+y_2) &= \int_a^b(y_1 + y_2)(y_1 + y_2)' \, dx \\
      &= \int_a^b y_1y_1' + y_1y_2' + y_2y_1' + y_2y_2' \, dx\\
      &= \underbrace{\int_a^b  y_1y_1'\,dx  + \int_a^b y_2y_2' \,
        dx}_{J(y_1) + J(y_2)} + \int_a^b
      y_1y_2' + y_2y_1'\, dx\\
      & \neq J(y_1) + J(y_2).      
    \end{split}
  \end{equation*}
  Thus, this is not linear.
\item $ J(y) = \int_a^b ((y')^2 + 2y) dx$.
\item[]Check additivity:
  \begin{equation*}
    \begin{split}
      J(y_1 + y_2) &= 
      \int_a^b ((y_1' + y_2')^2 + 2y_1 + 2y_2)\, dx\\
      & = \int_a^b y_1'y_1' + y_1'y_2' + y_2'y_1' + y_2'y_2' + 2y_1 +
      2y_2\, dx\\
      &= \underbrace{\int_a^b  y_1'y_1' + 2y_1\,dx  + \int_a^b
        y_2'y_2'  + 2y_2 \,dx}_{J(y_1) + J(y_2)} + \int_a^b
      y_1'y_2' + y_2'y_1'\, dx\\
    \end{split}
  \end{equation*}
Thus, this is not linear, either.
\item $ J(y) = e^{y(a)}$.
\item[]Check additivity:
  \begin{equation*}
    \begin{split}
      J(y_1 + y_2) &= e^{y_1(a) + y_2(a)}\\
      &= e^{y_1(a)}e^{y_2(a)}\\
      & \neq e^{y_1(a)} + e^{y_2(a)}.
    \end{split}
  \end{equation*}
Not linear.
\item $ J(y) = \int_a^b \int_a^b K(x,t) y(x) y(t)\, dx\, dt$.
\item[]Check additivity:
  \begin{equation*}
    \begin{split}
      J(y_1 + y_2) & = \int_a^b \int_a^b K(x,t) (y_1(x) + y_2(x))( y_1(t)
      + y_2(t))\, dx\, dt\\
      & = \int_a^b \int_a^b K(x,t) \left(y_1(x)y_1(t)  + y_2(x)y_1(t)
        + y_1(x)y_2(t) + y_2(x)y_2(t)\right)\, dx\, dt\\
      &=  \underbrace{\int_a^b \int_a^b K(x,t)y_1(x)y_1(t)\, dx\, dt  +
        \int_a^b \int_a^b K(x,t)y_2(x)y_2(t)\, dx\, dt}_{J(y_1) + J(y_2)}\\
      &\quad+ \int_a^b \int_a^b K(x,t)y_2(x)y_1(t)\, dx\, dt
        + \int_a^b \int_a^b K(x,t)y_1(x)y_2(t)\, dx\, dt
    \end{split}
  \end{equation*}
Thus, it's not linear.
\item $ J(y) = \int_a^b y \sin(x) dx$.
\item[]Check additivity:
  \begin{equation*}
    \begin{split}
      J(y_1 + y_2) & = \int_a^b (y_1 + y_2)  \sin(x) dx\\
      & = \int_a^b y_1 \sin(x) dx + \int_a^b y_2 \sin(x) dx\\
      & = J(y_1) + J(y_2).
    \end{split}
  \end{equation*}
Now, we can check the scaling property:
\begin{equation*}
  \begin{split}
    J(\alpha y) = \int_a^b (\alpha y) \sin(x) dx = \alpha \int_a^b y
    \sin(x) dx = \alpha J(y).
  \end{split}
\end{equation*}
Thus, we can say that this functional is indeed linear.
\item $ J(y) = \int_a^b (y')^2 dx + G(y(b)),$ where $G$ is a
  differential function.
\item[]Check additivity:
  \begin{equation*}
    \begin{split}
      J(y_1 + y_2) &= \int_a^b (y_1' + y_2')^2 dx + G(y_1(b) + y_2(b))\\
      & = \int_a^b y_1'(x)y_1'(t)  + y_2'(x)y_1'(t)
        + y_1'(x)y_2'(t) + y_2'(x)y_2'(t) dx + G(y_1(b) + y_2(b))
    \end{split}
  \end{equation*}
Compare this with 
\begin{equation*}
  \begin{split}
    J(y_1) + J(y_2) & = \int_a^b (y_1')^2 dx + G(y_1(b)) + \int_a^b (y_2')^2 dx + G(y_2(b)),
  \end{split}
\end{equation*}
and it can be seen that $G$ would definitely need to be linear for
$J$ to be linear, but we have no guarantee of that.  In addition, the
quadratic under the integral is \emph{ont} linear, so this functional
is not linear, either.
\end{enumerate}


\section*{Problem 7 (e)}
Compute the first variation of the functionals on $C^1[a,b]$ given in
Exercise 5(e)

\begin{enumerate}
\item[(e)] $ J(y) = \int_a^b y \sin(x) dx$.
\item[]\begin{equation*}
  \begin{split}
    \delta J(y_0, \alpha h) 
    &= \frac{d}{d\epsilon}J(y_0 + \epsilon h)|_{\epsilon
      =0}\\ 
    & = \frac{d}{d\epsilon}\left.\left(\int_a^b (y_0 + \epsilon h)
        \sin(x) dx\right)\right|_{\epsilon = 0}\\
    & =  \frac{d}{d\epsilon}\left.\left( 
      \int_a^b y_0 \sin(x) dx + 
      \epsilon \int_a^b  h \sin(x) dx\right)
      \right|_{\epsilon = 0}\\
    & =  \left.\left(\int_a^b  h \sin(x) dx \right)\right|_{\epsilon =
      0}\\
    & =  \int_a^b h \sin(x) dx\\
%    & = h\left(-\cos(b) + \cos(a)\right)
  \end{split}
\end{equation*}

\end{enumerate}


\section*{Problem 9}
The second variation of a functional $J: A \rightarrow \mathbb{R}$ at
$y_0 \in A$ in the direction $h$ is defined by 
\begin{equation*}
  \delta^2J(y_0,h) \equiv \frac{d^2}{d\epsilon^2}J(y_0 + \epsilon
  h)|_{\epsilon = 0}.
\end{equation*}
Find the second variation of the functional 
\begin{equation*}
  J(y) = \int_0^1 (xy'^2 + y \sin(y')) dx,
\end{equation*}
where $y \in C^2[0,1].$

\subsection*{My Answer}


\begin{equation*}
  \begin{split}
    J(y_0 + \epsilon h) & = \int_0^1 (x(y_0 + \epsilon h)'^2 + (y_0 +
    \epsilon h) \sin((y_0 + \epsilon h)')) dx \\
    & = \int_0^1 (x(y_0' + \epsilon h')^2 + (y_0 +
    \epsilon h) \sin(y_0' + \epsilon h'))\, dx \\
    &= \int_0^1 x(y_0'y_0' + 2\epsilon(y_o'h') + \epsilon^2(h')^2)
    dx 
    + \int_0^1 y_0\sin(y_0' + \epsilon h') dx
    + \epsilon\int_0 ^1 h\sin(y_0' + \epsilon h') dx \\
    &= \int_0^1 x(y_0')^2dx  + \int_0^1 y_0\sin(y_0' + \epsilon h')\, dx
    + \epsilon\left( 2\int_0^1 y_o' h'\, dx + \int_0 ^1 h\sin(y_0'
      + \epsilon h') dx\right) 
    + \epsilon^2\int_0^1 x(h')^2dx  
\end{split}
\end{equation*}
Now, Let's take the derivatives.
\begin{equation*}
  \begin{split}
   \frac{d}{d\epsilon}J(y_0 + \epsilon h) 
   &= \frac{d}{d\epsilon}\left[  \int_0^1 y_0\sin(y_0' + \epsilon h') dx
    + 2\epsilon \int_0^1 y_o' h'\, dx +\epsilon \int_0 ^1 h\sin(y_0'
      + \epsilon h')\, dx 
    + \epsilon^2\int_0^1x (h')^2dx  \right]\\
  &= 2\int_0^1 y_o' h'\, dx +
  \int_0^1 y_0\cos(y_0' + \epsilon h')\cdot h'\, dx
   +  \int_0 ^1 h\sin(y_0'
      + \epsilon h')\, dx\\
  &\quad  + \epsilon\left(\int_0^1 hh'\cos(y_0' +
      \epsilon h') dx\right) +2\epsilon \int_0^1 x(h')^2 dx 
  \end{split}
\end{equation*}

\begin{equation*}
  \begin{split}
     \frac{d^2}{d\epsilon^2}J(y_0 + \epsilon h) 
     & = \frac{d}{d\epsilon}\left[2\int_0^1 y_o' h'\, dx +
  \int_0^1 y_0\cos(y_0' + \epsilon h')\cdot h'\, dx
   +  \int_0 ^1 h\sin(y_0'
      + \epsilon h')\, dx \right.\\
  &\quad\left.  + \epsilon\left(\int_0^1 hh'\cos(y_0' +
      \epsilon h')\, dx\right) +2\epsilon \int_0^1x (h')^2\, dx\right] \\
  &= 0 - \int_0^1 y_0 (h')^2\sin(y_0' + \epsilon h') \,dx + \int_0^1
  hh'\cos(y_0' + \epsilon h')\,dx\\
  &\quad +\int_0^1 hh'\cos(y_0' +
      \epsilon h')\, dx - \epsilon \left(\int_0^1 h(h')^2\sin(y_0' +
      \epsilon h')\, dx\right) + 2\int_0^1x(h')^2\, dx.
  \end{split}
\end{equation*}
Setting $\epsilon = 0$, yields
\begin{equation*}
  \begin{split}
     \left. \frac{d^2}{d\epsilon^2}J(y_0 + \epsilon h)\right|_{\epsilon = 0} &=
 - \int_0^1 y_0 (h')^2\sin(y_0') \,dx + \int_0^1 hh'\cos(y_0')\,dx
 +\int_0^1 hh'\cos(y_0')\, dx + 2\int_0^1x(h')^2\, dx\\
 &= - \int_0^1 y_0 (h')^2\sin(y_0') \,dx + 2\int_0^1 hh'\cos(y_0')\,dx
+ 2\int_0^1x(h')^2\, dx\\
  \end{split}
\end{equation*}







\end{document}


= -\int_0^1 y_0(h')^2\sin(y_0' + \epsilon h')\, dx  + 2 \int_0^1
  hh' dx \\
  &\quad + \left(\int_0^1hy_0' + y_0h'\, dx +  \int_0 ^1 h\sin(y_0'
      + \epsilon h') dx\right)\left(-\int_0^1h(h')^2\sin(y_0' +
      \epsilon h')\, dx\right)\\
  &\quad + 
  \left(0 + \int_0^1hh'\cos(y_0' + \epsilon h')\, dx\right)
\left(\int_0^1 hh'\cos(y_0' +
      \epsilon h') dx\right)


%}